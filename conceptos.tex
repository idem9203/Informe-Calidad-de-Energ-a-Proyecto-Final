\chapter{Conceptos} \label{sec:Conceptos}

\section{Marco Teórico}

Un sistema eléctrico debería presentar comportamiento de parámetros eléctricos con las siguientes características: Amplitud uniforme, forma de onda sinusoidal, frecuencia constante, simetría y equilibrio entre las fases.

No obstante, la existencia de cargas no lineales como por ejemplo equipos electrónicos y dispositivos para el control del flujo de energía, hace que circulen corrientes no sinusoidal por la red, las cuales pueden ser consideradas como la superposición o suma de corrientes de di frentes frecuencia y múltiplos de la fundamental (Armónicas). Estas corrientes armónicas provocan caídas de tensión en la reactancia de cortocircuito, deformando la señal de tensión, con el consecuente efecto negativo sobre la operación normal de los componentes del sistema.

\section{Definiciones}

\textbf{Armónico:} Una componente sinusoidal de una onda o cantidad periódica que tiene una frecuencia que es un múltiplo entero de la frecuencia fundamental.

\textbf{Carga Critica:} Es aquella de cuyo funcionamiento incorrecto puede derivar en perjuicios económicos o de diversa índole, puede necesitar ser alimentada por fuentes de gran calidad.

\textbf{Carga Lineal:} Es aquella en donde la forma de onda de la corriente de estado estable, sigue la forma de onda de la tensión aplicada.

\textbf{Carga No Lineal:} Es aquella en donde la forma de onda de corriente de estado estable, no sigue la forma de onda de la tensión aplicada.

\textbf{Calidad de la Energía Eléctrica:} Es el grado de conformidad de las señales electromagnéticas, en un tiempo dado y en un nodo o punto definido, para cumplir con las necesidades de los consumidores, dentro del marco regulatorio del país.

\textbf{Calidad de la Potencia:} Conjunto de características de la electricidad en un punto dado de un sistema de potencia en un momento determinado, que permite satisfacer las necesidades requeridas por el usuario de la electricidad. Estas características son evaluadas con respecto a un conjunto de parámetros técnicos de referencia.

\textbf{Desbalance en Tensión o Corriente:} Es la máxima desviación de las tensiones o corrientes en un sistema trifásico del valor promedio. Es la relación de la componente de la secuencia negativa o cero a la componente de secuencia positiva, expresada en porcentaje.

\textbf{Distorsión:} Deformación de una señal (amplitud, frecuencia, fase) provocada por una perturbación.

\textbf{Factor de Distorsión:} Es la raíz cuadrada de la relación entre la suma de las amplitudes de todos los armónicos elevados al cuadrado y el cuadrado de la amplitud del fundamental.

\textbf{Factor de Potencia:} Relación entre la potencia activa (kW) y la potencia aparente (kVA) del mismo sistema eléctrico o parte de él.

\textbf{Frecuencia Fundamental:} frecuencia de la onda periódica original. En el caso de tensiones y corrientes de red esta es de 60Hz.

\textbf{Interferencia Electromagnética:} Degradación en las características de un dispositivo, equipo o sistema, causadas por una perturbación electromagnética.

\textbf{Orden de un Armónico (n):} Relación entre la frecuencia del armónico y la frecuencia fundamental.

\textbf{Perturbación Electromagnética:} Algún fenómeno electromagnético que puede degradar las características de desempeño de un dispositivo, equipo o sistema.

\textbf{Transitorio:} Designa un fenómeno o una cantidad que varia entre dos estados consecutivos durante un intervalo de tiempo corto comparado con la escala de tiempo de interés. Un transitorio puede ser un impulso unidireccional de cualquier polaridad o una oscilación abrupta en la onda con el primer pico ocurriendo en cualquier polaridad.