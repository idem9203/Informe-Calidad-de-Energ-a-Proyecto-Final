\chapter{Conclusiones y Recomendaciones} \label{sec:Conclusiones}

El sistema diseñado es apropiado para las características planteadas (portabilidad, tamaño, remoto, consumo, biocompatibilidad). Sin embargo, al momento de su implementación la principal complejidad de este fue la calibración de cada sensor debido a que cada sensor no cumplía con su curva característica la cual está definida por el fabricante como se observa en la figura \ref{fig:Curvas}. 

En cuanto al diseño electrónico se realizaron pruebas muy concisas que nos permitieron tener un bosquejo de las medidas aproximadas para cada sensor (figura \ref{fig:Comportamiento}), esto fue posible por las diferentes pruebas realizadas para que la ganancia del circuito sea apropiada para que abarque el rango de 1 a 5 Voltios.

El sistema presentado proporciona medidas simultaneas de presión de 5 sensores ubicados de la siguiente manera: sensor 1 corresponde al dedo grueso, sensor 2 y 3 al primer y quinto metatarsiano respectivamente, el sensor 4 en la zona lateral de la planta y por ultimo el sensor 5 ubicado en el talón. El análisis de los datos se puede llevar a cabo con la aplicación web que gráfica las variaciones de presión en tiempo real, de esta forma el sujeto obtiene una retroalimentacion inmediata sobre su patología la cual puede ser corregida en el mismo instante.

Este sistema puede ser utilizado en diferentes tipos de aplicaciones como pueden ser la elección de zapatos mas apropiados, rehabilitación de diferentes lesiones, estudio de patrones a la hora de caminar por personas afectadas por enfermedades como Parkinson y Diabetes.
