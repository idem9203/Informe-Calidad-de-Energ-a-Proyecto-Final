\chapter{Introducción} \label{sec:Introduccion}
Los sistemas electrónicos para el monitoreo del rendimiento deportivo basados en redes de área corporal, emplean sensores para adquirir las señales del deportista al realizar el gesto técnico y un dispositivo para transmitir esta información a un computador o aplicación móvil, que luego es procesada y analizada, Se pueden usar para cuantificar el rendimiento y determinar las técnicas óptimas \cite{mertz2013technology}, \cite{waltz2015quantified}. En Colombia, el uso de herramientas tecnológicas en el deporte no está muy difundido. En particular, en levantamiento de pesas, las ejecuciones incorrectas de los movimientos disminuyen el rendimiento y puede causar lesiones. Este documento describe el diseño y construcción de un sistema electrónico para el monitoreo de la presión plantar durante la ejecución de ejercicios de levantamiento de pesas. El sistema está basado en un arreglo de sensores de fuerza ubicados en la planta del pie, que permiten monitorear la presión plantar en tiempo real. Por medio de este sistema puede detectarse puntos de alta presión, estimar el centro de presión (COP) del pie, evaluar la simetría del gesto técnico y detectar pronación excesiva del pie.

El estudio de la presión plantar estática y dinámica en el movimiento del pie es importante en el campo de la salud y la actividad deportiva para la toma de medidas correctivas \cite{keijsers2013classification, hills2001plantar, ostadabbas2014knowledge, adelsberger2014effects}. Una medición precisa y confiable de los parámetros del movimiento permiten clasificar los patrones de la marcha como normales, o patológicos, y evaluar su eficiencia. En la actualidad hay sistemas de medición plantar disponibles en el mercado, pero a un alto precio. 

%Corregido
La halterofilia consiste en levantar una barra con discos de diferentes pesos. Existen dos modalidades de competición: arranque (snatch), en la cual se debe elevar la barra sin interrupción desde el suelo hasta extender los brazos sobre la cabeza. Y dos tiempos (clean and jerk), que permite una interrupción del movimiento a la altura de los hombros, para después elevar la barra hasta la extensión completa de brazos. En ambas modalidades la ubicación de los pies es esencial para la óptima ejecución del ejercicio, porque definen la postura del atleta.

Entre las aplicaciones más comunes de las tecnologías para el registro de la presión plantar está el análisis de la marcha para el tratamiento de patologías y corrección de movimientos. Se emplean plataformas de fuerza y sistemas de análisis de movimiento en 3-D (M3D) que miden la reacción de las fuerzas triaxiales del suelo (GRF) y orientaciones 3-D de los pies \cite{liu2012mobile}. En particular, este método se basa en una combinación de cámaras de alta velocidad para la captura de las orientaciones de los segmentos corporales en 3-D de los pacientes. Sin embargo, las aplicaciones de estos dispositivos estacionarios están restringidas a la investigación experimental, y es difícil encontrar aplicaciones de análisis en ambientes cotidianos o críticos como lo establece la práctica de un deporte de riesgo.\\

Con respecto al levantamiento de pesas, en el trabajo de Liu and Chen se capturó el plano sagital del atleta a través de cámaras de alta resolución y se usaron plantillas “EMED Pedar” para la adquisición de los datos sobre presión plantar \cite{liu2001foot}.

Recientemente, Martinez et.al. \cite{martinez2014embedded} desarrollaron un sistema inalámbrico para medir la presión plantar en cuatro puntos anatómicos. Un sistema comercial en el zapato es F-scan (Tekscan Inc., Boston, MA, EE. UU.). Puede medir la presión plantar hasta 862 kPa con alta resolución espacial. Otro sistema es OpenGo (Moticon, Munich, Alemania) que incluye 13 sensores con alcance de hasta 400 kPa y comunicación inalámbrica. Aunque estos sistemas ofrecen una alta resolución espacial, tienen un alcance reducido que puede no ser apropiado para levantamiento de pesas.

%Corregido
Existen diversas condiciones médicas en las cuales se requiere analizar la presión plantar. En el artículo “A portable insole plantar pressure measurement system” \cite{wertsch1992portable} se plantea la construcción de un sistema portable de una plantilla de presión para ser usado en actividades cotidianas como lo son caminar, trotar y correr. Los sensores se distribuyeron en puntos críticos de la plantilla y el sistema es capaz de recolectar información durante 2 horas continuas sin interferir en las actividades cotidianas.

Los dispositivos para la medición de la presión plantar también se han utilizado en deportes como baloncesto, atletismo y tenis. En el artículo “Biomechanical analysis of the plantar and upper pressure with different sports shoes” se describe un estudio en el cual se midieron la presión plantar y la presión dorsal en diferentes tipos de calzado \cite{mei2014biomechanical}. Se usó un sistema de plantilla con sensores de presión en 4 zonas superiores. Como resultado de este estudio la diferencia de los zapatos en el área anterior-posterior y media-lateral no resulta significativa debido a que los zapatos son del mismo material con la misma suela, mientras que la dureza y blandura si son factores importantes para el resultado.

En este trabajo se describe el desarrollo de una plantilla instrumentada con sensores de fuerza para el monitoreo de la presión plantar para aplicación en deportes. El sistema ofrece un rango amplio (mayor a 6 MPa), transmisión inalámbrica y una aplicación web para visualización y análisis de datos. 