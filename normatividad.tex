\chapter{Normatividad} \label{sec:Normatividad}

\textbf{Tensión:} De \textbf{Resolución CREG 024/2005 ANEXO 1} Las tensiones en estado estacionario a 60Hz no podrían ser inferiores al 90\% de la tensión nominal ni ser superiores al 110\% de esta durante un periodo superior a un minuto. Esta variable eléctrica depende del operador de red en este caso ELECTRICARIBE.

\textbf{Frecuencia: Resolución CREG 070/98 6.2.1.1} La frecuencia nominal del SIN es 60Hz y su rango de variación de operacion esta entre 59.8 y 60.2 Hz en condiciones normales de operación. El OR y los Usuarios deben tener en cuenta que en estados de emergencia, fallas, déficit energético y periodos de restablecimiento, la frecuencia puede oscilar entre 57.5 y 63.0 Hz por un periodo de tiempo de quince (15) segundos, en concordancia con lo establecido en los numerales 2.2.5 y 5.1 del Código de Operación incluido en el Código de Redes (Resolución CREG 025 de 1995).

\textbf{Distorsión Armónica de Tensión (THD):} La magnitud de armónicos admisible en un sistema se encuentra establecido, entre otros, por la norma \textbf{IEEE Standard 519-1992}, "IEEE Recommended Practices and Requierements for Harmonic Control in Power Systems", fijando los limites para la distorsión de tensión según la tensión nominal del sistema:

\begin{table}[H]
\small
%\footnotesize
%\scriptsize
%\tiny
\begin{center}
\begin{tabular}{| l | c | c |}
 \hline
             \multicolumn{3}{|c|}{\textbf{LIMITES DE DISTORSIÓN DE TENSIÓN}}              \\
 \hline
    Tensión en el PCC    & Distorsión de tensión Individual & Distorsión total de tensión \\  
 \hline
    Menor de 69 kV       &                 3                &              5              \\
 \hline
	Entre 69 kV y 161 kV &                1,5               &             2,5             \\
 \hline
	Mayor de 161 kV      &                 1                &             1,5             \\
 \hline
\end{tabular}
\caption{Limites de Distorsión de Tensión.}
\label{Table:Limites_Distor_Tension}
\end{center}
\end{table}

\textbf{Limites de distorsión en Corriente:} Las corrientes armónicas para cada usuario son evaluadas en la acometida y los limites se establecen en base a la relación entre la corriente de cortocircuito y la demanda máxima de corriente de la carga del usuario.

\textbf{Distorsión Armónica de Corriente (TDD):} La magnitud de armónicos admisible en un sistema.

\begin{table}[H]
\small
%\footnotesize
%\scriptsize
%\tiny
\begin{center}
\begin{tabular}{| c | c | c | c | c | c | c |}
 \hline
                                            \multicolumn{7}{|c|}{\textbf{IEEE 519}}                                                \\
 \hline
                  \multicolumn{7}{|c|}{\textbf{Limites de la distorsión armónica en corriente en la acometida}}                  \\
 \hline
    $I_{cc}/I_L$ &    TDD    &  $h < 11$   &   $11 \leq h < 17$   &   $17 \leq h < 23$   &   $23 \leq h < 35$   &   $h \geq 35$  \\ 
 \hline
                                           \multicolumn{7}{|c|}{\textbf{$V_n \leq 69 kV$}}                                       \\
 \hline
    $< 20$       &  $5.0\%$  &    $4.0\%$  &         $2.0\%$      &        $1.5\%$       &         $0.6\%$      &     $0.3\%$    \\
 \hline
	$20-50$      &  $8.0\%$  &    $7.0\%$  &         $3.5\%$      &        $2.5\%$       &         $1.0\%$      &     $0.5\%$    \\
 \hline
	$50-100$     & $12.0\%$  &   $10.0\%$  &         $4.5\%$      &        $4.0\%$       &         $1.5\%$      &     $0.7\%$    \\
 \hline
    $100-1000$   & $15.0\%$  &   $12.0\%$  &         $5.5\%$      &        $5.0\%$       &         $2.0\%$      &     $1.0\%$    \\
 \hline
    $>1000$      & $20.0\%$  &   $15.0\%$  &         $7.0\%$      &        $6.0\%$       &         $2.5\%$      &     $1.4\%$    \\
 \hline
                                       \multicolumn{7}{|c|}{\textbf{$69 kV < V_n \leq 161 kV$}}                                  \\
 \hline
    $< 20*$      &  $2.5\%$  &    $2.0\%$  &         $1.0\%$      &       $0.75\%$       &         $0.3\%$      &     $0.15\%$   \\
 \hline
	$20-50$      &  $4.0\%$  &    $3.5\%$  &        $1.75\%$      &       $1.25\%$       &         $0.5\%$      &    $0.25\%$    \\
 \hline
	$50-100$     &  $6.0\%$  &    $5.0\%$  &        $2.25\%$      &        $2.0\%$       &        $0.75\%$      &    $0.35\%$    \\
 \hline
    $100-1000$   &  $7.5\%$  &    $6.0\%$  &        $2.75\%$      &        $2.5\%$       &         $1.0\%$      &     $0.5\%$    \\
 \hline
    $>1000$      & $10.0\%$  &    $7.5\%$  &         $3.5\%$      &        $3.0\%$       &        $1.25\%$      &     $0.7\%$    \\
 \hline
                                              \multicolumn{7}{|c|}{\textbf{$V_n > 161 kV$}}                                      \\
 \hline
    $< 50$       &  $2.5\%$  &    $2.0\%$  &         $1.0\%$      &       $0.75\%$       &         $0.3\%$      &     $0.15\%$   \\
 \hline
	$\geq 50$    &  $4.0\%$  &    $3.5\%$  &        $1.75\%$      &       $1.25\%$       &         $0.5\%$      &    $0.25\%$    \\
 \hline
\end{tabular}
\caption{Limites de Distorsión armónica de corriente.}
\label{Table:Limites_Distor_Corriente}
\end{center}
\end{table}

\begin{itemize}
    \item * Todos los equipos de generación de energía están limitados a estos valores de corriente, sin importar la relación ${I}_{cc}/I_L$.
    \item Para las armónicas pares, los limites son el $25\%$ de los valores específicos en la tabla.
    \item No se permite la existencia de componentes de corriente directa, que corresponden a la armónica cero.
    \item Si las cargas que producen las armónicas utilizan convertidores con numero de pulsos q mayor a 6, los limites indicados en la tabla se incrementan por un factor (Ver Ecuación \ref{eqn:incremento_factor}). 
\end{itemize}

\begin{equation} \label{eqn:incremento_factor}
    \sqrt{\frac{q}{6}}
\end{equation}

La Distorsión de demanda total TDD esta definida como (Ver Ecuación \ref{eqn: Ecuacion_TDD}):

\begin{equation} \label{eqn: Ecuacion_TDD}
\small
    TDD = \frac{\sqrt{\sum\limits_{h=2}^{\infty}{{I_h}^{2}}}}{I_L} \times 100\%
\end{equation}

Donde:

$I_h:$ Magnitud de la armónica individual.

$I_L:$ Demanda máxima de la corriente fundamental de la carga.

$h:$ Orden armónico impar.

${I}_{cc}:$ Deben utilizarse aquella que bajo condiciones normales de operación, resulte en la mínima corriente de cortocircuito en la acometida, ya que este valor reduce la relación ${{I}_{cc}}/{I_L}$ y la evaluación es mas severa.

$I_L:$ Es la demanda máxima de la corriente fundamental en la acometida y puede calcularse como el promedio de las demandas máximas de corriente mensuales de los ultimo 12 meses o puede estimarse para usuarios que inician su operación.



\section{Fuentes que producen las Armónicas}

La norma IEEE 519-1992, relativa a "Practicas recomendadas y requerimientos para el control de armónicos en sistemas eléctricos de potencia" agrupa a las fuentes emisoras de armónicas en tres categorías diferentes:

\begin{itemize}
    \item Dispositivos electrónicos de potencia.
    \item Dispositivos productos de arcos eléctricos.
    \item Dispositivos ferromagnéticos.
\end{itemize}

Algunos de los equipos y procesos que se ubiquen en estas categorías son:

\begin{itemize}
    \item Motores de corriente directa accionados por tiristores.
    \item Inversores de frecuencia.
    \item Fuentes ininterrumpidas UPS.
    \item Computadoras.
    \item Equipo electrónico.
    \item Hornos de inducción.
    \item Equipos de soldadura.
    \item Transformadores sobreexcitados.
\end{itemize}



\section{Efectos de las Armónicas}

Las corrientes armónicas generadas por cargas no lineales, están desfasadas noventa grados con respecto al voltaje que las produce, fluyendo una potencia distorsionante de la fuente a la red eléctrica y viceversa, que solo es consumida como perdidas por efecto Joule que se transforman en calor, de forma equivalente a la potencia reactiva fundamental relacionada al factor de potencia de desplazamiento.

Algunos de los efectos nocivos producidos por el flujo de corrientes armónicas son:

\begin{itemize}
    \item Aumento en las perdidas por efecto Joule (${I^2} \times R$).
    \item Sobrecalentamiento en conductores del neutro.
    \item Sobrecalentamiento en motores, generadores, transformadores y cables, reduciendo su vida.
    \item Vibración en motores y generadores.
    \item Falla de bancos de capacitores.
    \item Falla de transformadores.
    \item Efectos de resonancia que amplifican los problemas mencionados anteriormente y pueden provocar incidentes eléctricos, mal funcionamiento y fallos destructivos de equipos de potencia y control.
    \item Problemas de funcionamiento en dispositivos electrónicos sensibles.
    \item Interferencias en sistemas de telecomunicaciones.
\end{itemize}

Los efectos dependerán de la proporción que exista entre la carga no lineal y la carga total del sistema, aunado a que se debe mantener la distorsión dentro de los limites establecidos por las normas.